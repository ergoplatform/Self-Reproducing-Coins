The script works in a following way. Every output contains its bit value $val$,
column index $x$, and the length of the grid $n$ at the current step. As the
grid expands by one at every step, $n$ also serves as the row number. By
default, the transaction spends three inputs (corresponding to the three
neighboring bits from the previous row), and creates three outputs with the same
bit value by the automaton rule. One output flagged by $mid$ is supposed to be
spent for new value with the column number $x$, and another two --- for the
columns $x\pm 1$ (see Fig.~\ref{fig:txBit}). In case the transaction creates the
boundary cells, either one or two inputs are emulated to have zero bit values
(lines 20---32).  The overall validation script checks the correctness of the
positions of inputs (lines 12 and 13) and outputs (line 17), correspondence of
of the bit values (line 15), the correctness of the $mid$  flag assignment for
inputs (line 10) and the fact that all outputs are identical except the $mid$
flag, which is set only once (lines 2---7).
