\documentclass[runningheads]{llncs}
\usepackage{graphicx}
\usepackage{color}
\usepackage{algorithm}
\usepackage{algpseudocode}
\graphicspath{{./figures/}}
\bibliographystyle{splncs04}

\usepackage{nameref}
\usepackage{hyperref}
%\usepackage{amsmath}
\newcommand{\knote}[1]{\textcolor{green}{A: {#1}}}
\newcommand{\dnote}[1]{\textcolor{red}{D: {#1}}}
\newcommand{\vk}[1]{\textcolor{blue}{V: {#1}}}
\newcommand{\payload}{\textit{payload}}
\newcommand{\script}{\textit{script}}
%\newcommand{\And}{\&\&}
\newcommand{\And}{\textsf{ and }}
%\newcommand{\And}{\wedge}
\def\Let#1#2{\State #1 $\gets$ #2}
\algnewcommand{\IfThenElse}[3]{\algorithmicif\ #1\ \algorithmicthen\ #2\
\algorithmicelse\ #3}
\algnewcommand\Or{\textsf{or }}
\algnewcommand\Xor{\textsf{ xor }}
\algnewcommand\Int{\textsf{int}}
\algnewcommand\Not{\textsf{not}}
\algnewcommand\Mod{\textsf{ mod }}

\begin{document}
    \title{Self-replicating UTXOs as a universal Turing Machine}

\author{Alexander Chepurnoy\inst{1,2}, Vasily Kharin\inst{3}, Dmitry Meshkov\inst{1}}

\institute{Ergo Platform \\\email{catena@protonmail.com} \and
IOHK Research \\\email{alex.chepurnoy@iohk.io} \and
Research Institute\\\email{v.kharin@protonmail.com}}

    \date{\today}
    \maketitle

    \begin{abstract}
        %        Обычно Тьюринг-полнота языков смаркт контрактов блокчейне
        % систем связывается с многообразмем возможностей языка, в
        % частности с наличием циклов.
        %        Мы показываем, что даже если в языке явным образом отсутствуют
        % циклы, возможно добиться Тьюринг полноты, если разворачивать
        % рекурсивные вызовы не внутри одной транзакции, а между ними
        %        ??? у системы есть возможность задать зависимость между
        % выходным и входным состояниями транзакции.
        Turing completeness of the smart contract languages in the blockchain
        systems is often associated with the variety of the language features
        (such as loops). %within a context of a single transaction%
		A more fundamental problem is combining unpredictability of arbitrary % Тут не сразу понятно, чья это проблема (это проблема циклов - непонятное время исполнения)
        script execution time with the blockchain requirements.
        We show that Turing completeness in the blockchain environment can
        be achieved through unwinding the recursive calls between
        multiple transactions and blocks instead of using a single one. We prove
        it constructing the simple universal Turing machine using
        the small set of features of the scripting language in the unspent
        transaction output (UTXO) model, which admits the relations between
        input and output transaction states.
        Neither unbound loops nor infinite validation time is needed in this approach.

    \keywords{smart contracts, Turing completeness, blockchain, cellular automata}
    \end{abstract}

    \section{Introduction}
    %    История о блокчейне - Сатоши и т.д. Биткоин скрипт
    %    Биткоин скрипт начали развивать, утверждая что ему не хватает оператора ГОТО для Тьюринг полноты, обзор дискуссий по Тьюринг-полноте смарт контрактов (возможно тут тезис Черча-Тьюринга). Пачка статей, которые говорят про полноту, но никто не доказывал. 
    %    История о Тьюринг полноте вообще - несколько неформальных определений, тезис Черча-Тьюринга, и то как доказывается тьюринг полнота. Простейший пример - правило 110.
    %    Наш вклад - доказательство тьюринг полноты в UTXO блокчейне без оператора ГОТО через реализацию правила 110.
    %    Структура статьи
    Blockchain technology has become widely adopted after the introduction of
    Bitcoin by S.~Nakamoto~\cite{nakamoto2008bitcoin}. This peer-to-peer
    electronic cash ledger drew the enormous attention from the public, which
    resulted in rapid development of the technology and appearance of hundreds
    of alternative cryptocurrency projects. It also turned out that the
    blockchain applications expand quite far beyond the simple ledger niche. The
    rules of transaction validation can incorporate complicated logic, which is
    the essence of so-called smart contracts. In the case of Bitcoin this logic
    is implemented in the special purpose Script language, which is believed not
    to be Turing complete. This belief stimulated development of other smart
    contract platforms with the emphasis on the language universality.
    Particularly, in Ethereum~\cite{buterin2014next} the $jump$ opcode was
    introduced in the transaction scripting language in order to incorporate
    unlimited loops.  In practice this resulted to various vulnerabilities and
    DoS attacks \cite{atzei2017survey} since transaction computation cost
    (so-called $gas$) can only be calculated in runtime.
    Moreover, Turing-completeness of Ethereum system is
    still a subject of debates mostly due to undecidability of the halt problem
    in combination with the bounded block validation time. Particularly, the gas
    limit is often viewed at as a fundamental component preventing
    Turing-completeness~\cite{miller2016ethereum}.
    %    Their approach, while being successful in general, exposed few
    %    significant complications both from formal and security
    %    sides~\cite{atzei2017survey,momot2016seven}.

    Turing complete programming language is the language which allows
    description of the universal Turing machine. The universal Turing machine is
    the Turing machine which can simulate any other Turing machine; its
    existence is one of the main results of the Turing
    theory~\cite{turing1937computable}. The study of Turing machines is strongly
    motivated by the Church---Turing thesis, which states that Turing machines
    are capable of universal computation. This thesis is often viewed at as a
    definition of computation and computability~\cite{turing1939systems}. The
    set of known computation devices and models was rapidly growing during the
    twentieth century, and the methods of their analysis were improved as well.
    The usual way of proving the Turing completeness of a system of rules,
    a device or a language is using it to emulate the system that is already
    proven to be Turing complete.  The system extensively used in our work is a
    one-dimensional cellular automaton called Rule 110. It was conjectured to
    be Turing complete by S.~Wolfram~\cite{wolfram1986theory}. The conjecture
    was proven by M.~Cook~\cite{cook2004universality} based on the previous
    works by E.~L.~Post~\cite{post1943formal}.

    The utter internal simplicity of Rule 110 makes it an appealing system for
    the proofs of Turing completeness. In the present work we prove Turing
    completeness of the whole class of the smart contract languages by explicit
    construction of the Rule 110. We do not require neither loops, nor jump
    operator, nor recursive calls inside the transaction. Instead, we treat the
    computation as if it is occurring between the transactions (or maybe
    blocks). In this context transaction chaining and replication furnishes us
    with potentially infinite loops and recursion, while the union of outputs
    inside multiple transactions yields analog of a potentially infinite tape.
    The underlying idea of complexity growth is similar to the one expressed
    in~\cite{von1951general}.

    This paper is structured as follows: in the Section \ref{section2} we first
    describe a naive implementation of Rule 110 using simple Bitcoin-like
    scripting language.  Then we discuss the pitfalls arising from compliance
    with the blockchain properties, and show the way to overcome them.  Section
    \ref{section3} contains the discussion of the implementation, language
    requirements, and usability of the method for the real-world blockchain
    applications. \nameref{appendix1} sketches the discussion on the nature of
    computation in the framework of blockchain scripting and validation rules.

    \section{Rule 110 implementation}
    \label{section2}
    %    Из чего будем строить машину, Описание нужного куска языка Эрго (возможно это тоже в аппендикс)
    %    Код кидаем в аппендикс, тут описываем его логику. Описываем логику преобразования одной строки.
    %    Картинки как с блоками меняется состояние автомата 110
    %    В блокчейне необходимо иметь константное время валидации для выполнения свойства chain growth (GKL). Для Тьюринг полноты нужно уметь работать со строками произвольного размера, чтобы их валидацуия выполнялась за конечное время, из нужно разбивать на подстроки, крайний случай - 1 бит на выход. (ссылка на код либо в аппендикс, либо на гитхаб)
    Cellular automaton we construct is known as Rule 110.  It is a
    one-dimensional string of 0s and 1s together with evolution rules.  The one
    step evolution of the bit is defined by its value $c$ together with the
    values of the two neighboring bits --- the left $\ell$ and the right $r$
    with the transition rule defined in Alg.~\ref{alg:calc_bit}
    \begin{algorithm}[H]
\caption{Transition function of the Rule 110 automaton}
\label{alg:calc_bit}
\begin{algorithmic}[1]
    \Function{calcBit}{$\ell$, $c$, $r$}
    \State
    \Return $\ell c r + cr + c + r$ mod 2
    \EndFunction
\end{algorithmic}
\end{algorithm}


    For the automaton implementation in a blockchain we use Bitcoin-like
    transactions consisting of inputs and outputs. Every output consists of a
    protecting \script{} and a \payload{} while input is a reference to an
    output from a previous transaction.  We assume that the current state of the
    automaton is stored in the transaction output's \payload{}.  The general
    idea is to use the next transaction as a single step of the system
    evolution. In order to achieve this, two main conditions must be satisfied.
    First, the \payload{} of at least one newly generated output should contain
    the updated state of the automaton. Second, this output must contain exactly
    the same script. These conditions require the transaction input to have
    access to the output's \script{}s and \payload{}s.  It is implicitly present
    in the vast amount of existing blockchains, since in most cases scripts
    verify the signature of the spending transaction, which is constructed over
    the byte array containing the new outputs.  However, this way of access to
    an output's data may be hardly exploitable.  In this section we assume that
    spending script has a direct access to the outputs.

    Keeping all these in mind, we come to the following validation
    script structure:
        %\begin{algorithm}[H]
    %    \caption{isRule110 function that checks, that transformation from
    %    inLayer to outLayer is correct}
    %    \label{alg:isRule110}
    %    \begin{algorithmic}[1]
    %        \Function{\sf isRule110}{inLayer, outLayer}

    %        \Function{\sf calcBit}{$i$}
    %        \Let{$\ell$}{
    %            \IfThenElse{$i==0$} {inLayer[inLayer.size$-1$]} {inLayer[$i-1$]}
    %        }
    %        %if($i$ == 0) inLayer[inLayer.size] else inLayer[$i-1$]}
    %        \Let{$c$}{inLayer[$i$]}
    %        %\Let{$r$}{if($i$ == inLayer.size) inLayer[0] else inLayer[$i+1$]}
    %        \Let{$r$}{
    %            \IfThenElse{$i==inLayer.size-1$} {inLayer[0]} {inLayer[$i+1$]}
    %        }
    %        \If{   $\ell$ == 0 \And{} c == 0 \And{} r == 0} \Return 0
    %        \ElsIf{$\ell$ == 0 \And{} c == 0 \And{} r == 1} \Return 1
    %        \ElsIf{$\ell$ == 0 \And{} c == 1 \And{} r == 0} \Return 1
    %        \ElsIf{$\ell$ == 0 \And{} c == 1 \And{} r == 1} \Return 1
    %        \ElsIf{$\ell$ == 1 \And{} c == 0 \And{} r == 0} \Return 0
    %        \ElsIf{$\ell$ == 1 \And{} c == 0 \And{} r == 1} \Return 1
    %        \ElsIf{$\ell$ == 1 \And{} c == 1 \And{} r == 0} \Return 1
    %        \ElsIf{$\ell$ == 1 \And{} c == 1 \And{} r == 1} \Return 0
    %        \EndIf
    %        \EndFunction
    %        \State \Return outLayer == inLayer.indeces.map(calcBit)
    %        \EndFunction
    %        \vskip8pt
    %    \end{algorithmic}
    %\end{algorithm}

    %\begin{algorithm}[H]
    %    \caption{isRule110 function that checks, that transformation from
    %    inLayer to outLayer is correct}
    %    \label{alg:isRule110mod3}
    %    \begin{algorithmic}[1]
    %        \Function{\sf isRule110}{inLayer, outLayer}

    %        \Function{\sf calcBit}{$i$}
    %        \Let{$\ell$}{
    %            \IfThenElse{$i==0$} {inLayer[inLayer.size$-1$]} {inLayer[$i-1$]}
    %        }
    %        \Let{$c$}{inLayer[$i$]}
    %        \Let{$r$}{
    %            \IfThenElse{$i==inLayer.size-1$} {inLayer[0]} {inLayer[$i+1$]}
    %        }
    %        \Let{$x$}{4$\ell+$2c$+$r}
    %        \State
    %        \Return \Int(\Not($x == 0$ \Or $x == 4$ \Or $x == 7$))
    %        \EndFunction
    %        \State 
    %        \Return outLayer == inLayer.indeces.map(calcBit)
    %        \EndFunction
    %        \vskip8pt
    %    \end{algorithmic}
    %\end{algorithm}

    %\begin{algorithm}[H]
    %    \caption{isRule110 function that checks, that transformation from
    %    inLayer to outLayer is correct}
    %    \label{alg:isRule110mod2}
    %    \begin{algorithmic}[1]
    %        \Function{\sf isRule110}{inLayer, outLayer}
    %        \Function{\sf calcBit}{$i$}
    %        \Let{$\ell$}{inLayer[$(i-1)$ \% inLayer.size]}
    %        \Let{$c$}{inLayer[$i$]}
    %        \Let{$r$}{inLayer[$(i+1)$ \% inLayer.size]}
    %        \Let{$x$}{4\Int($\ell$)$+2$\Int($c$)$+$\Int($r$)}
    %        \State
    %        \Return \Not($x == 0$ \Or $x == 4$ \Or $x == 7$)
    %        \EndFunction
    %        \State \Return outLayer == inLayer.indeces.map(calcBit)
    %        \EndFunction
    %        \vskip8pt
    %    \end{algorithmic}
    %\end{algorithm}

    %\begin{algorithm}[H]
    %    \caption{isRule110 function that checks, that transformation from
    %    inLayer to outLayer is correct}
    %    \label{alg:isRule110mod}
    %    \begin{algorithmic}[1]
    %        \Function{\sf isRule110}{inLayer, outLayer}
    %        \Function{\sf calcBit}{$i$}
    %        \Let{$\ell$}{inLayer[$(i-1)$\Mod inLayer.size]}
    %        \Let{$c$}{inLayer[$i$]}
    %        \Let{$r$}{inLayer[$(i+1)$\Mod inLayer.size]}
    %        \State
    %        \Return $c$\Xor$r$\Xor$(c$\And$r)$\Xor$(\ell$\And$c$\And$r)$
    %        \EndFunction
    %        \State \Return outLayer == inLayer.indeces.map(calcBit)
    %        \EndFunction
    %        \vskip8pt
    %    \end{algorithmic}
    %\end{algorithm}

\begin{algorithm}[H]
\caption{[Description]}
\label{alg:isRule110}
\begin{algorithmic}[1]
    \Function{validate}{in, out}
    \Function{isRule110}{inLayer, outLayer}
    \Function{procCell}{$i$}
    \Let{$\ell$}{inLayer[$i-1$ mod inLayer.size]}
    \Let{$c$}{inLayer[$i$]}
    \Let{$r$}{inLayer[$i+1$ mod inLayer.size]}
    \State
    \Return \Call{calcBit}{$\ell$, $c$, $r$}
    \EndFunction
    \State \Return outLayer == inLayer.indeces.map(\textsc{procCell})
    \EndFunction
    \State
    \Return \Call{isRule110}{in[0].payload, out[0].payload}
    $\wedge$ (in[0].script = out[0].script)
    \EndFunction
\end{algorithmic}
\end{algorithm}

    This script performs two checks. First, it takes the payload of a current
    input and ensures, that the application of Rule 110 to it equals to the
    payload of the first output. Second, it checks that the first output script
    is the same as a script of the input. The full implementation of this script
    in the smart contract language of an existing UTXO blockchain Ergo is
    provided at \cite{ergoScript1}.

    With this script, the cellular automaton evolution may be started by
    chaining transactions in a blockchain. Fig.~\ref{fig:txs} shows three
    transactions (on the left), each one representing the iteration of the
    automaton (on the right).

    \begin{figure}[h]
        \centering
        \includegraphics[width=.8\textwidth]{row_tx.pdf}
        \caption{Transaction chain following Rule 110. See
            Alg.~\ref{alg:isRule110} for the
            $script$ field description.
        \label{fig:txs} }
    \end{figure}
    Potentially infinite number of sequential transactions in the blockchain
    leads to the potentially infinite evolution of cellular automaton which is
    necessary for Turing complete systems.  However, there is one pitfall left.
    The size of the data stored in the output must be limited from above, since
    the general blockchain rules~\cite{garay2015bitcoin} require the transaction
    validation time to be bound from above in order to satisfy the chain growth
    property.

    The workaround is to split the automaton state between
    transactions once it becomes too large. As an extreme case one can make a
    transaction output play a role of a single bit of the automaton. While being
    inefficient this implementation keeps the logic simple and complies with the
    requirements of the blockchain and of potentially infinite evolution in
    space and time. The pseudocode of the corresponding script is
    provided in the Alg.~\ref{alg:txBit} and it's implementation in Ergo smart
    contract language is provided at \cite{ergoScript2}. Fig.~\ref{fig:bit_txs}
    schematically shows the sequence of transactions (on the left), that corresponds
    to some area evaluation (on the right) of the automaton run.
    \begin{figure}[h]
        \centering
        \includegraphics[width=.8\textwidth]{bit_tx_greedy.pdf}
        \caption{Evolution of the cellular automaton described in
            Alg.~\ref{alg:txBit}. Every non-boundary transaction spends three
            outputs, and generates three new ones with identical bit values.
            Hatching indicates ``mid'' flag being unset. Numbers in the cells on
            the right pane correspond to the transaction numbers on the left.
			% Тут матрица транзакций транспонирована относительно правой картинки - трудно читать. 
			% К тому же на картинке выделен блок шириной в 5 клеток, хотя показана эволюция только трех. 
			% Понятно, что крайние две тоже участвуют в определении состояния, но поначалу сбивает с толку. 
			% Идеально на клеточках справа проставить номера соответсвующих им транзакций
        \label{fig:bit_txs} }
    \end{figure}
    %The logic of the script is the following: every output in its $payload$
%contains the following data: $val$ --- the value of the corresponding bit;
%$x$ --- bit's column; $n$ --- number of columns in the current row; 
%$mid$ --- the flag explained below. Every transaction creates three
%outputs, which are the replicas of the same cell with the $val$ computed by Rule
%110. Replication is needed in order
%to feed the value to the three cells in the next row. The leftmost cell emulates
%the virtual zero input. So far nothing prevents using 9 outputs of the three neighboring
%transactions to create 9 outputs for three identical cells in the next row. To
%deal with it we introduce flag $mid$, which is $true$ for one of three outputs
%only, stating that only this output must be used as a middle bit for the next
%row.
%
%With this being said, we have the following things to verify in the script:
%correctness of the input $x$, $y$, $mid$, correctness of the output $x$, $y$,
%$mid$, $n$, script replication in the outputs, and compliance of the output $val$
%with the Rule 110.
\begin{algorithm}[H]
    \caption{[Description]}
    \label{alg:txBit}
    \begin{algorithmic}[1]
        \Function{verify}{in, out}\Comment{``in'' and ``out'' are lists of inputs and outputs}
        \Function{calcBit}{$\ell$, $c$, $r$}
        \State
        \Return $\ell c r + cr + c + r$ \Mod 2
        \EndFunction
        \Function{properlyReplicated}{out}
        \Let{isCopy1}{out[1] = out[0].copy(mid$\leftarrow true$)}
        \Let{isCopy2}{out[2] = out[0].copy(mid$\leftarrow false$)}
        \State
        \Return ($\neg$out[0].mid) $\wedge$ isCopy1 $\wedge$ isCopy2
        \EndFunction
        \Function{correctPayload}{in, out}
        \Let{inMidCorrect}{in[1].mid $\wedge$ $\neg$(in[0].mid $\vee$ in[2].mid)}
        \Let{inYCorrect}{(in[0].n = in[1].n) $\wedge$ (in[0].n = in[2].n)}
        \Let{inXCorrect}{(in[1].x = in[0].x$+1$) $\wedge$ (in[1].x = in[2].x$-1$)}
        \Let{inCorrect}{inXCorrect $\wedge$ inYCorrect $\wedge$ inMidCorrect}
        \Let{valCorrect}{out[0].val=\Call{calcBit}{in[1].val, in[0].val, in[2].val}}
        \Let{outPosCorrect}{out[0].x = in[0].x $\wedge$ (out[0].n = in[0].n$+1$)}
        \Let{outCorrect}{valCorrect $\wedge$ outPosCorrect $\wedge$
        \Call{properlyReplicated}{out}}
        \State
        \Return outCorrect $\wedge$ inCorrect $\wedge$ in.size=3 $\wedge$ out.size=3
        \EndFunction
        \If{in[0].x=n $\wedge$ in.size=1}
        \Let{middle}{in[0].copy(x$\leftarrow$n$-1$, val$\leftarrow$0, mid$\leftarrow true$)}
        \Let{left}{in[0].copy(x$\leftarrow$n$-2$, val$\leftarrow$0, mid$\leftarrow false$)}
        \State
        realIn = left ++ middle ++ in
        \ElsIf{in[0].x=n $\wedge$ in.size=2}
        \Let{left}{in[0].copy(x$\leftarrow$n$-1$, val$\leftarrow$0, mid$\leftarrow false$)}
        \State
        realIn = left ++ in
        \Else
        \State
        realIn = in
        \EndIf
        \State
        \Return \Call{correctPayload}{realIn, out} $\wedge$
        out[0].script=in[0].script
        \EndFunction
    \end{algorithmic}
\end{algorithm}


	The script works as follows. Every output's payload contains its bit value
    $val$, 
    the column index $x$, and the length of the grid $n$ at the current step.
    As the
    grid expands by one at every step, $n$ also serves as the row number. By
    default, the transaction spends three inputs (corresponding to the three
    neighboring bits from the previous row), and creates three outputs with the same
    bit value by the automaton rule. One output flagged by $mid$ is supposed to be
    spent for new value with the column number $x$, and another two --- for the
    columns $x\pm 1$ (see Fig.~\ref{fig:bit_txs}). In case the transaction creates the
    boundary cells, either one or two inputs are emulated to have zero bit values
    (lines 20--32).  The overall validation script checks the correctness of the
    positions of inputs (lines 12 and 13) and outputs (line 17), correspondence of
    of the bit values (line 15), the correctness of the $mid$  flag assignment for
    inputs (line 10) and the fact that all outputs are identical except the $mid$
    flag, which is set only once (lines 2--7).

    Since the Turing-completeness of the Rule 110 was proven
    in~\cite{cook2004universality}, we conclude that even though the scripting
    language itself does not allow infinite loops, Turing-completeness of the
    system can be achieved by combining multiple transactions together. Note
    that our language requirements are not very demanding, and can be met in a 
    number of platforms avoiding complicated ad-hoc structures.

    \section{Discussion}
    \label{section3}
    %    Описываем, что мы сделали рекурсию через вызов выходом самого себя, хотя в самом языке нет циклов и рекурсии.
    %    Внутриннюю логику можно усложнять для сокращения количества тактов, что приводит к space-time tradeoff - либо пишешь гигантский скрипт который выполняется за 1 такт, либо скрипт простой, но нужно много тактов.
    %    Правило 110 только это только доказательство полноты, в реальности все делается гораздо эффективнее т.к. в почти любом языке смарт контрактов есть арифметика, не нужно пытаться реализовывать логику через клеточные автоматы.
    %    Показать что это не экзотика, а достежимо в других языках, например язык Waves и Ethereum (+ посмотреть, что еще есть).
    %    Практические выводы: 1) позволяет ответить на вопрос, какой класс задач возможно реализовать в данном языке смарт контрактов. 2) Открывает возможность писать тяжелые смарт контраты, требуемые вычисления которых превышают лимит вычислений на блок, 3) ???
    %    Где же происходят вычисления? ибо мы не задаем алгоритм, а задаем правила верификации. Т.к. в заголовке мы обозначили другой результат, тут обозначить тему и вынести в аппендикс 2
    The most important move in our work was unwinding recursive calls by means
    of transaction chaining, although the language contained neither cycles nor
    recursion. By doing this we let the program be executed over the set of
    transactions and blocks. This in turn helped us to work around the halting
    problem, which used to make arbitrary script execution impossible inside a
    transaction in the view of the chain growth property. In our approach a single
    transaction approximately corresponds to a single cycle of the computation
    machine. The trade-off is again between the consumed space and execution
    time: one can either incorporate complex logic in the script, and expect
    relatively fast execution, or leave the script logic short and simple, but
    the execution can consume more transactions.

    It worth noting that any practical algorithm implementation can be
    significantly optimized using the language built-ins such as arithmetic
    operations. Our construction is not needed for that, it just
    guarantees that any algorithm can be potentially implemented and executed.
    We also claim that our approach can be generally used for Turing completeness
    proofs of various smart contract languages, e.g. it might be possible to prove
    that smart contracts of Waves platform \cite{wavesSmarts} are actually
    Turing-complete, although the authors claimed the opposite.

    Blockchain strict upper-bound for transaction validation time makes unbounded
    loops useless in terms of variety of algorithms that may be implemented
    inside one transaction.
    In absence of unbounded loops one can estimate script complexity in compile-time
    and divide algorithm into multiple transactions if needed. Thus, the hard
    computation limit per block may be surpassed by chaining multiple transactions
    between different block.

    As a drawback, one can argue that the set of validation rules is not exactly
    the programming language, since it doesn't prescript the sequence of actions,
    but rather the way to check whether the result is correct. Here we stay on the
    practical side: in our case the validation description is detailed enough to
    convert it to the strict sequence of actions, hence it actually defines an
    algorithm. For the more detailed discussion on the subject we address reader
    to \nameref{appendix1}.

    \bibliography{sources}
    \section{Appendix~1}
    \label{appendix1}

    %    Где же происходят вычисления? ибо мы не задаем алгоритм, а задаем правила верификации. Можно сравнение, что в обычном компьютере движущая сила это электричество.. Оффчейн вычисления - требуем найти корни многочлена, или дискретный логарифм.. Посмотреть на декларативные языки программирования, может какие-нидь клеточные автоматы в экселе. Pay-for-proof contracts из биткоина.

\end{document}
